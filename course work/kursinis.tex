\documentclass{VUMIFPSkursinis}
\usepackage{algorithmicx}
\usepackage{algorithm}
\usepackage{algpseudocode}
\usepackage{amsfonts}
\usepackage{amsmath}
\usepackage{bm}
\usepackage{caption}
\usepackage{color}
\usepackage{float}
\usepackage{graphicx}
\usepackage{listings}
\usepackage{subfig}
\usepackage{url}
\usepackage{wrapfig}
\usepackage[table,xcdraw]{xcolor}
\usepackage[backend=biber]{biblatex}

% Titulinio aprašas
\university{Vilniaus universitetas}
\faculty{Matematikos ir informatikos fakultetas}
\department{Programų sistemų katedra}
\papertype{Kursinis darbas}
\title{Didelių duomenų srautų analizė, anomalijų aptikimas}
\titleineng{Big data analysis, detection of anomalies}
\status{3 kurso 3 grupės studentas}
\author{Jokūbas Rusakevičius}
\supervisor{dr. Vytautas Valaitis}
\date{Vilnius – \the\year}

% Nustatymai
\setmainfont{Palemonas}

\bibliography{bibliografija}

\begin{document}
\maketitle

\tableofcontents

\sectionnonum{Įvadas}
Šis darbas yra Programų Sistemų studijų trečio kurso privalomas kursinis darbas apie anomalijų aptikimą didžiuosiuose duomenyse (angl. \textit{Big Data}) bei anomalijų aptikimą naudojant atviro kodo analitinį įrankį „MacroBase“.

\subsectionnonum{Problematika}
Remiantis 2001 metais „Gartner“ pateiktu apibrėžimu (iki šiol laikomu pagrindiniu (angl. \textit{go-to})), didieji duomenys - tai duomenys, kurie turi didelę įvairovę (angl. \textit{variety}), yra renkami nuolat didėjančiais kiekiais (angl. \textit{volumes}) ir generuojami vis didėjančiais greičiais (angl. \textit{velocity}), šis apibrėžimas dar vadinamas tryjų „V“ \cite{three_vs}. Paprastai, didžiuosius duomenis galima apibrėžti kaip duomenų rinkinius tokius didelius, kad tradiciniai programiniai duomenų apdorojimo įrankiai nesugeba jų sugauti, tvarkyti, organizuoti, apdoroti ar su jais dirbti priimtiname laiko intervale \cite{bigdata_tolerable_time}, jie yra paprasčiausiai per dideli ir per daug sudėtingi. Tačiau šie duomenys turi milžinišką potencialą ir gali būti panaudojami sprendžiant verslo ir kitas problemos, kurių sprendimas iki šiol buvo neįmanomas.\par

Generuojant ir saugant milžiniškus kiekius duomenų, natūraliai, užfiksuojami tokie duomenys, kurie išsiskiria ir yra nebūdingi duomenų rinkiniui. Duomenų vienetai kurie yra nukrypę nuo kitų duomenų rinkinyje yra vadinami anomalijomis. Anomalijų aptikimas yra procesas, kurio metu yra aptinkama ir identifikuojama anomalija arba išskirtis (angl. \textit{outlier}) duomenų rinkinyje \cite{anomaly}. Anomalijos yra retas reiškinys, tačiau jų egzistavimas gali reikšti didelį pavojų taikomajai sistemai ar jos naudotojams. Šio darbo metu bus tiriamas anomalijų aptikimas remiantis iš „Backblaze“ 2018 metų (pirmą ketvirtį) duomenų centruose esančių diskų surinkta informacija \cite{backblaze}.\par
Dėl įvairių priežasčių renkamų, saugomų ir operuojamų duomenų kiekiai nuolatos didėja ir netgi gerokai lenkia žmogaus sugebėjimą juos apdoroti ar analizuoti. Didžiosios socialinių tinklų kompanijos Twitter, Facebook ir LinkedIn 2015-2016 metais pranešė kiekviena atskirai fiksuojanti iki 12 milijonų įvykių per sekundę \cite{twitter, facebook, linkedin}. Taip pat negalima pamiršti vis labiau plintančių ir didelius kiekius duomenų generuojančių automatizuotų duomenų šaltinių („Dalykų Interneto“ (angl. \textit{Internet of Things} arba \textit{IoT})). Be to, tokie palankūs veiksniai kaip kylantis automatizuotų duomenų šaltinių populiarumas, pinganti techninė įranga, išvystyti komunikaciniai tinklai bei mažėjančios duomenų saugojimo kainos paskatino dešimčių milijardų dolerių komercines investicijas šių technologijų vystimui \cite{iot_investments}. Dėl šių ir kitų veiksnių numatoma, kad kiekvienais metais bendras duomenų kiekis išaugs po 40\% \cite{iot}, o iki 2020 metų pranašaujama, kad bendras pasaulinis duomenų kiekis peržengs 40 zetabaitų ($4\times{10}^{22}$ baitų) ribą \cite{future_data_volume}.\par

Didžiųjų duomenų laikomos informacijos paslėpta nauda ir svarba yra visuotinai pripažįstama, tačiau ši informacija nėra lengvai išgaunama. Duomenų peržiūra ir analizė sudaro labai didelį krūvį tiek analitikui, tiek analitiniams įrankiams. Fizinė duomenų peržiūra yra paprasčiausiai neįmanoma, o nuolatos didėjantys duomenų kiekiai vis labiau atskiria dėmesio reikalaujančius duomenis ir ribotą dėmesį turintį analitiką. Net ir aukščiausios kvalifikacijos analitikai praneša panaudojantys vos iki 6\% jų surenkamų duomenų \cite{prioritizing_attention}. Todėl atsiranda iššūkis prioritizuoti žmogaus dėmesį. Nors žmogui yra neįmanoma peržiūrėti visų šių duomenų, tačiau kompiuteriai ir/ar mašinos gali. Informacinės sistemos labiau nei bet kada turi filtruoti, akcentuoti, jungti, grupuoti pateiktus duomenis, jiems suteikti kontekstą ir rodyti naudotojui tik ribotą, svarbią bei apibendrintą informaciją. Visa rodoma, bet nereikalinga informacija reikalauja ir eikvoja žmogaus dėmesį \cite{attention}.\par

Standfordo universitetas kartu su Masačusetso technologijos institutu 2017 metais paskelbė kuriantys naują atviro kodo, ne tik didžiųjų, bet ir greitųjų duomenų (angl. \textit{Fast Data}) analitinį paieškos įrankį. „MacroBase“ pagrindinis uždavinys yra žmogaus dėmesio prioritizavimas. Vienas iš „MacroBase“ šio uždavinį sprendimų yra sugeneruoti didžiausio dėmesio reikalaujančią supaprastintą išvestį, kurios neįprastus duomenų vienetus „MacroBase“ padeda aiškinti pagal duomenų atributus \cite{macrobase_overview, prioritizing_attention}. „MacroBase“ yra naujas ir modernus analitinis įrankis, pateikiantis inovatoriškų sprendimų vis didėjančių ir greitėjančių duomenų srautų analizei atlikti bei galintis grąžinti tikslius rezultatus dirbdamas 2 milijonų įvykių per sekundę greičiu per užklausą per branduolį, dėl to, šiame darbe bus plačiau nagrinėjamas bei eksperimentai atliekami naudojant būtent šį įrankį.

\subsectionnonum{Darbo tikslas ir uždaviniai}
Šio darbo \textbf{tikslas} - palyginti iki šiol naudotas didelių duomenų analizavimo technologijas ir pasinaudoti „MacroBase“ duomenų analizavimo ir anomalijų aptikimo įrankį anomalijų aptikimui.\par

Darbui iškelti \textbf{uždavyniai}:\par

\begin{enumerate}
\item Paaiškinti kas yra „Didieji Duomenis“.
\item Surasti ir palyginti dabar naudojamus anomalijų dideliuose duomenyse aptikimo įrankius.
\item Išanalizuoti „MacroBase“.
\item Paruošti eksperimentui reikalingą įrašų rinkinį iš „Backblaze“ kiekvieną ketvirtį skelbiamų duomenų.
\item Įsidiegti ir paruošti darbui „MacroBase“ analitinį įrankį.
\item Atlikti eksperimentus ir aptikti anomalijas paruoštuose duomenyse.
\item Pateikti galutines eksperimento išvadas.
\end{enumerate}

\section{Didieji duomenys}
Pasak MIT Media Lab direktoriaus Joi Ito, duomenų užrašymas ant popieriaus lapo dar visai neseniai buvo vienintelis būdas rinkti informaciją. Žmogus sugalvotas idėjas ir mintis užrašydamas ant popieriaus lapo jas paversdavo žiniomis. Tačiau dabartinė didžiųjų duomenų situacija yra kitokia. Priešingai nei seniau, surenkamų duomenų kiekiai yra milžiniški, tačiau jie nėra žinios, tol kol jų nepradedama nagrinėti ir analizuoti. Tik pradėjus analizuoti duomenis atliekant su jais įvairias transformacijas, galima pastebėti, kad gaunama įdomi ir netgi svarbi informacija\cite{documentary}. \par

Terminas „Didieji duomenys“ vartojamas apibūdinant procesą naudojamą, kai tradicinės duomenų rinkimo (angl. \textit{data mining}) ir tvarkymo metodikos nebegali įžvelgti ar atskleisti duomenyse esančios prasmės \cite{bigdata}. Nors pirmieji dideli duomenų įrašų rinkiniai yra datuojami jau 1960-1970 metais \cite{first_bigdata}, pats terminas buvo pradėtas naudoti tik nuo 1990 metų, o jo autoriumi yra laikomas ar bent už termino išgarsinimą yra dėkojama John Mashley \cite{john_1, john_2, john_3}. Tačiau kaip didžiųjų duomenų „dydis“ nuolatos plečiasi ir kinta, taip ir jų apibrėžimas, bei apibrėžimui naudojamų „V“ kiekis (\ref{subsec:požymiai} poskyris).\par

Didieji duomenys apima visų trijų struktūrizacijos lygių duomenis: struktūrizuotus, iš dalies struktūrizuotus ir nestruktūrizuotus, tačiau dižiausias didžiųjų duomenų dėmesys yra skiriamas nestruktūrizuotiems duomenims \cite{bigdata_data_struct}. Šie skirtingo struktūrizacijos lygio duomenys pasižymi tokiomis savybėmis:

\begin{itemize}
\item \textbf{Struktūrizuoti duomenys} - tai duomenys, laikomi jiems skirtuose fiksuotuose laukuose, failuose ar įrašuose. Struktūrizuoti duomenys priklauso nuo jiems priskirto duomenų modelio, kuris nurodo kokie, kokio tipo (skaitiniai duomenys, valiutos, alfabetiniai duomenys, vardai, datos, adresai) ir kokių apribojimų (simbolių skaičius, terminų apribojimas, pvz.: Ponas, Ponia, Prof., Doc. ir kt) laukai bus saugomi. Šio tipo duomenys yra lengvai operuojami, pridedami ir analizuojami. Dėl didelių duomenų saugojimo kaštų tai  ilga laiką buvo vienintelis sprendimas duomenų saugojimui - viskas, ko negalima optimaliais struktūrizuoti, laikoma popieriniame formate \cite{structured_data}.

\item \textbf{Iš dalies struktūrizuoti duomenys} \cite{semistructured_data} - tai tarpinis variantas tarp struktūrizuotų ir nestruktūrizuotų duomenų. Duomenys neturi jiems priskirtos formalios duomenų modelio struktūros susietos su duomenų bazėmis ar kitomis duomenų lentelių formomis, tačiau turi žymes ar kitas žymėjimo priemones įrašų atskyrimui ir įrašų hierarchijos sukūrimui. Todėl iš dalies struktūrizuoti duomenys yra dar žinomi kaip save apibūdinančios (angl. \textit{self-describing}) struktūros. Iš dalies struktūrizuotuose duomenyse įrašai gali priklausyti vienai klasei ir būti grupuojami kartu, tačiau turėti skirtingus atributus, o pati atributų išdėstymo tvarka nėra svarbi. XML ir kitos žymių kalbos yra iš dalies struktūrizuotų duomenų pavyzdžiai \cite{structured_data}.

\item \textbf{Nestruktūrizuoti duomenys} - tai duomenys, kuriems skiriamas didžiausias didžiųjų duomenų dėmesys. Šio tipo duomenys neturi jiems priskirto duomenų modelio ar taisyklių pagal, kurias jie būtų organizuojami, ir jų neišeina klasifikuoti pagal žmogui skaitomus požymius. Tai daugiaprasmiai duomenys, kurių sudėti tvarkingai į vieną „dėžutę“ yra neįmanoma: nuotraukos, paveikslėliai, vaizdo įrašai, transliuojami instrumentiniai duomenys, PDF dokumentai, pristatymų skaidrės, elektroniniai laiškai, tinklaraščiai, teksto dokumentai, knygos, medicinos įrašai ir kt. Nors kiekvienas prieš tai išvardytas duomenų tipas atskirai yra struktūrizuotas, tačiau jų rinkinys vis tiek laikomas nestruktūrizuotais duomenimis \cite{structured_data}. Yra apskaičiuota, kad apie 80\% visų duomenų yra nestruktūrizuoti \cite{unstructured}.
\end{itemize} 

Toliau šiame skyriuje bus aprašomi didžiuosius duomenis apibūdinantys penki „V“ (\ref{subsec:požymiai} poskyris. Galiausiai bus rašoma apie didžiųjų duomenų pritaikymą (\ref{subsec:pritaikymas} poskyris).

\subsection{Didžiųjų duomenų didieji „V“} \label{subsec:požymiai}
Originaliai 2001 metais pateiktas didžiųjų duomenų apibrėžimas susidėjo iš trijų „V“: įvairovės (angl. \textit{variety}), kiekio (angl. \textit{volume}) ir greičio (angl. \textit{velocity}) \cite{three_vs}. Tačiau per paskutinius kelis metus šis apibrėžimas buvo papildytas dviem naujais „V“: duomenų teisingumu (angl. \textit{veracity}) ir verte (angl. \textit{value}) \cite{five_vs_everyone_must_know}. 2016 metais pateiktas atnaujintas didžiųjų duomenų apibrėžimas: didžiuosius duomenis sudaro informacijos rinkiniai charakterizuojami tokių aukštų duomenų kiekių, greičių ir įvairovės, kad yra reikalingos specifinės technologijos ir analitiniai metodai vertės iš tos informacijos išgavimui \cite{bigdata_16}. Toliau šiame poskyryje bus aprašomas kiekvienas „V“ (Kiekis (\ref{subsubsec:kiekis} punktas), greitis (\ref{subsubsec:greitis} punktas), įvarovė (\ref{subsubsec:įvairovė} punktas), teisingumas (\ref{subsubsec:teisingumas} punktas) ir vertė (\ref{subsubsec:vertė} punktas)) atskirai.

\subsubsection{Keikis - „Volume“} \label{subsubsec:kiekis}
\textbf{Kiekis} nusako didžiulį kiekį duomenų, kurie yra sugeneruojami kiekvieną sekundę. „FaceBook“ kiekvieną dieną sugeneruoja 4 petabaitus ($1 PB = {10}^{15} B$) naujų duomenų, vien mygtukas „Patinka“ yra paspaudžiamas virš 4 milijonų kartų per minutę, o naujų kiekvieną dieną įkeliamų nuotraukų skaičius siekia net 350 milijonų \cite{facebook_stats}. Sugeneruojamų duomenų kiekis per minutę beveik prilygsta visiems iki 2008 metų surinktiems duomenims. Tačiau pasinaudojus didžiųjų duomenų technologija ir išskirstytą (angl. \textit{distributed}) sistemą, kur duomenų dalys sujungtos programinės įrangos yra laikomos skirtingose vietose, yra įmanoma juos saugoti ir jais naudotis.

\subsubsection{Greitis - „Velocity“} \label{subsubsec:greitis}
\textbf{Gretis} nusako greitį, kuriuo yra generuojami nauji duomenys, ir greitį kuriuo duomenys yra perduodami. „CERN“ „LHC“ dalelių greitintuvas generuoja 1 megabaitą neapdorotų duomenų per įvykį, o per vieną sekundę yra užregistruojami net 600 milijonų įvykių (iš viso 600 TB/s) \cite{cern}. Didžiųjų duomenų technologijos sudaro sąlygas analizuoti duomenis, kol jie yra generuojami taip niekada nepatalpinant jų į duomenų bazę.

\subsubsection{Įvairovė - „Variety“} \label{subsubsec:įvairovė}
\textbf{Įvairovė} nusako skirtingus duomenų tipus. Dauguma duomenų dabar, ne taip kaip praeityje, yra nestruktūrizuoti, kas padaro juos sunkiai patalpinamus į duomenų lenteles. Pasinaudojus didžiųjų duomenų technologija galima sujungti skirtingų rūšių duomenis kartu su tradiciniais struktūrizuotais duomenimis.

\subsubsection{Teisingumas - „Veracity“} \label{subsubsec:teisingumas}
\textbf{Teisingumas} nusako duomenų netvarkingumą ir patikimumą. Dėl skirtingų didžiųjų duomenų formų, kokybė ir tikslumas yra beveik nevaldomi. Tačiau didieji duomenys ir analitinės technologijos sugeba dirbti su šio tipo duomenimis, dažnai kompensuodamos duomenų netikslumą ir nekokybiškumą dideliais duomenų kiekis.

\subsubsection{Vertė - „Value“} \label{subsubsec:vertė}
\textbf{Vertė} nusako iš didžiųjų duomenų gaunamą naudą ir dažnai yra laikomas svarbiausiu iš visų „V“. Yra svarbu dirbant su didžiaisiais duomenimis paversti juos į tam tikra vertę. Tai nebuvo vienas iš originaliai apibrėžtų „V“, tačiau buvo įvardintas dėl didelių darbo su didžiaisiais duomenimis kaštų ir nevisada aiškios gaunamos naudos.

\subsection{Pritaikymas} \label{subsec:pritaikymas}
Didžiuosius duomenis pritaikymas yra galimas beveik bet kokioje srityje, kur yra generuojami dideli kiekiai duomenų. Didžiųjų duomenų industrija tokia didelė, kad 2010 metais ji buvo verta apie 100 milijardų dolerių ir augo po beveik 10\% per metus \cite{bigdata_industry}. Didieji duomenys yra populiarūs, nes iš jų gaunamos žinos yra plačiai pritaikomos. Todėl didžiųjų duomenų pritaikymo pavyzdžių šiais laikais galima rasti visur:
\begin{itemize}
\item Gamintojai gauna vieną didžiausių naudų. Naudojant didžiuosius duomenis gamintojai gali tobulindami savo tiekimo planus, gaminti pagal numatytą paklausą ir pasiekti beveik nulinį tiekimo laiką.
\item Medicina didieji duomenys yra naudojami siekiant pritaikyti suasmenintą mediciną \cite{in_medicine}.
\item Edukacinės institucijos kaip universitetai pradėjo kurti su duomenų analize susijusias studijų programas 2011 metais paskelbtai 1,5 milijonų aukštos kvalifikacijos duomenų analitikų trūkumo \cite{shortage} paklausai patenkinti.
\item Žiniasklaida naudoja naudoja didžiųjų duomenų analizę pateikdamos suasmenintas, kryptingas reklamas.
\item Kt.
\end{itemize}
Su vis didėjančiais duomenų kiekiais ir generavimo greičiais didėja ir duomenyse esančių žinių potencialas. Remiantis 2014 metų duomenimis, vienas trečdalis visų duomenų yra saugomi skaitiniu ir/ar raidiniu (angl. \textit{alphanumeric}) tekstų arba nuotraukų pavidalu \cite{one_third}, o tai formatas naudingiausias duomenų pritaikymui. Taip pat, didelis potencialas slypi didžiųjų duomenų analizėm nenaudojamuose vaizdo ir audio bei kituose duomenyse.

\section{Didžiųjų duomenų analizavimas}


\section{„MacroBase“ analitinis įrankis}


\section{Duomenų analizės eksperimentas naudojant „MacroBase“ analitinį įrankį}
Eksperimentui atlikti buvo pasirinktas „MacroBase GUI“ analitinio įrankio „MacroBase“ grafinė naudotojo sąsaja. Į „MacroBase GUI“ eksperimento metu buvo žiūrima kaip į „juodąją dėžę“.  CSV tipo failai buvo pasirinktas duomenų šaltinio tipas atliekant eksperimentą.

\subsection{Duomenų rinkinys}
Šiame poskyryje aprašyti eksperimentui pritaikyti „Backblaze“ standžiųjų diskų stebėjimo duomenys \cite{backblaze}.

\subsubsection{Eksperimentui pritaikyti duomenys}
Eksperimentui bus naudojami „Backblaze“ duomenų centruose esančių standžiųjų diskų atliekamo darbo stebėjimų duomenys. „Backblaze“ nuo 2013 metų kiekvieną ketvirtį paviešina surinktus duomenis stebint jų duomenų centruose esančius standžiuosius diskus. Kiekvieno ketvirčio duomenys yra pateikiami CSV failais - vienas CSV failas vienai dienai. Vidutinis vienos dienos įrašų eilučių skaičius faile - 100 tūkst; dydis - 28MB. Eksperimentui buvo pasirinktas, paskutinis paviešintas, 2018 metų pirmo ketvirčio duomenų rinkinys \cite{backblaze} - 90 CSV failų.

\subsubsubsection{Duomenų aprašymas}
Kiekvienas „Backblaze“ duomenų centrų dienos įrašų CSV failas yra sudarytas iš duomenų, kurių didžiąją dalį sudaro „S.M.A.R.T.“ \cite{smart_meaning} atributai:

\begin{itemize}
\item \textit{\textbf{Date}} - įrašo įrašymo data, užrašoma yyyy-mm-dd formatu (dėl grupavimo į failus pagal dienas, visi įrašai viename faile turės tą pačią datą).
\item \textit{\textbf{Serial Number}} - gamintojo priskirtas kietojo disko serijos numeris.
\item \textit{\textbf{Model}} - gamintojo priskirtas kietojo disko modelio numeris.
\item \textit{\textbf{Capacity Bytes}} - disko dydis baitais.
\item \textit{\textbf{Failure}} - įvykusios klaidos žymėjimui skirtas atributas. „0“, jei kietasis diskas dirba korektiškai; „1“, jei tai buvo paskutinė diena, kai diskas buvo naudojamas prieš sugesdamas.
\item \textit{\textbf{SMART}} atributai - 100 stulpelių duomenų iš kurių 50 yra neapdorotų ir 50 normalizuotų duomenų stulpelių (atributų reikšmės yra aprašytos prieduose (Priedas \ref{sec:smart})).
\end{itemize}

\subsubsubsection{Eksperimentui atrinkti naudingi SMART atributai}
Didelė dalis „Backblaze“ duomenų rinkinio įrašų atributų nėra naudingi atliekamam eksperimentui. Todėl buvo pasirinkti keli naudingi atributai atmetant, tokius atributus kurie yra specifiniai gamintojams arba nėra naudinigi be disko konteksto. Buvo palikti trivialūs atributai nurodantys disko temperatūrą ir bendrą valandinį disko darbo laiką, bei 5 atributai, kuriuos „Backblaze“ naudoja numatyti diskų sugedimus \cite{backblaze_failure}: perskirstytų sektorių skaičiaus, neištaisomų klaidų skaičiaus, sustabdytų dėl per ilgo darbo laiko komandų skaičiaus, „nestabilių“ sektorių skaičiaus, nepataisomų sektorių skaičiaus. Šios reikšmės atitinka šiuos atributus:
\begin{itemize}
\item \textit{\textbf{Smart 5 Raw}} - perskirstytų sektorių skaičius;
\item \textit{\textbf{Smart 9 Raw}} - bendras valandinis disko darbo laikas;
\item \textit{\textbf{Smart 187 Raw}} - neištaisomų klaidų skaičius;
\item \textit{\textbf{Smart 188 Raw}} - sustabdytų dėl per ilgo darbo laiko komandų skaičius;
\item \textit{\textbf{Smart 194 Raw}} - disko temperatūra laipsniais Celcijaus;
\item \textit{\textbf{Smart 197 Raw}} - „nestabilių“ sektorių skaičius;
\item \textit{\textbf{Smart 198 Raw}} - nepataisomų sektorių skaičius;
\end{itemize}

\subsubsubsection{Duomenų paruošimas} \label{subsubsubsec:duomenų_paruošimas}
Eksperimentas atliktas naudojant individualius dienų failus, tačiau dėl didesnio duomenų kiekio faile bei bendros analizės buvo paruošti 4 papildomi CSV failai: kiekvienam mėnesiui po vieną ir viso ketvirčio bendras. Tam buvo parašytas BASH skriptas, kuriam per argumentą yra pateikiama „Backblaze“ CSV failų direktorija. Skriptas pavadintas „combine\_csv.sh“; skriptas grąžina failus pavadinimais: „combine\_all.csv“ (su bendrais duomenimis), „combine\_01.csv“ (su sausio mėnesio duomenimis), „combine\_02.csv“ (su vasario mėnesio duomenimis), „combine\_03.csv“ (su kovo mėnesio duomenimis). Skriptas yra pateiktas prieduose (Priedas \ref{sec:skriptas}).
\begin{itemize}
\item Bendras CSV failas \textbf{„combine\_all.csv“}:
\begin{itemize}
\item Dydis: \textbf{2,5GB};
\item Įrašų eilučių skaičius: \textbf{$8 949 492$};
\end{itemize}

\item Sausio mėnesio CSV failas \textbf{„combine\_01.csv“}:
\begin{itemize}
\item Dydis: \textbf{846MB};
\item Įrašų eilučių skaičius: \textbf{$3 039 306$};
\end{itemize}

\item Vasario mėnesio CSV failas \textbf{„combine\_02.csv“}:
\begin{itemize}
\item Dydis: \textbf{782,3MB};
\item Įrašų eilučių skaičius: \textbf{$2 803 852$};
\end{itemize}

\item Kovo mėnesio CSV failas \textbf{„combine\_03.csv“}:
\begin{itemize}
\item Dydis: \textbf{868,2MB};
\item Įrašų eilučių skaičius: \textbf{$3 106 334$}.
\end{itemize}
\end{itemize}

\subsection{Eksperimentui naudotos aplinkos aprašimas}
Eksperimentas buvo atliekamas naudojant „Ubuntu (64-bit)“ operacinę sistemą įdiegtą virtualioje mašinoje „Oracle VM VirtualBox“. „MacroBase GUI“ įdiegtas ir paruoštas darbui naudojantis „MacroBase“ dokumentaciją.

\subsubsection{Virtuali mašina eksperimentui}
„MacroBase“ pateikiami pavyzdžiai ir konfigūraciniai nurodymai yra pateikiami „Linux“ operacinėms sistemoms. Todėl eksperimentui atlikti buvo pasirinkta atviro kodo nemokama operacinė sistema „Ubuntu“. Dėl paprastumo buvo nuspręsta operacinei sistemai naudoti virtualią mašiną. Atviro kodo nemokama virtuali mašina „Oracle VM VirtualBox“ buvo pasirinkta šiai užduočiai. Galutinės eksperimentui naudotos sisteminės specifikacijos:
\begin{itemize}
\item „Ubuntu“ operacinės sistemos 64 bitų versija \textbf{„Ubuntu (64-bit)“}, versija: \textbf{16.04 LTS};
\item \textbf{„Oracle VM VirtualBox“}, versija: \textbf{5.2.12} r122591 (Qt5.6.2);
\item Virtualus standusis diskas: \textbf{40GB};
\item Virtualiai mašinai skirta operatyvioji atmintis: \textbf{4GB}.

\end{itemize}

\subsubsubsection{Virtualios mašinos paruošimas}
Virtualios mašinos paruošimas darbui:
\begin{enumerate}
\item  Iš „Oracle VM VirtualBox“ internetinės svetainės atsisiunčiamas naujausias „Windows 10“ (operacinė sistema į kuria diegiama virtuali mašina) operacinę sistemą palaikantis diegimo failas (\url{https://www.virtualbox.org/wiki/Downloads}).
\item Sekant sąrankos vedlio nurodymus įdiegiama „Oracle VM VirtualBox“ virtuali mašina.
\item Iš „Ubuntu“ internetinės svetainės atsisiunčiamas naujausios „Ubuntu (64-bit)“ operacinės sistemos ISO failas (\url{https://www.ubuntu.com/download/desktop}).
\item Atidarius „Oracle VM VirtualBox“ programinę įrangą pradedamas naujos operacinės sistemos pridėjimas spaudžiant ant mygtuko su tekstu „Nauja“.
\item Toliau rodomuose languose: pasirenkamas operacinės sistemos tipas - „Linux“; versija: „Ubuntu (64-bit)“; nurodomas operatyviosios atminties kiekis megabaitais: 4096MB; sukuriamas virtualus standusis diskas: 40GB.
\item „Oracle VM VirtualBox“ pagrindiniame lange pasirinkus naujai sukurtą virtualią mašiną spaudžiama ant mygtuko su tekstu „Paleisti“.
\item Atsiradusiame virtualios mašinos lange pasirenkamas prieš tai atsiųstas „Ubuntu (64-bit)“ ISO failas.
\item Pasirenkama „Install“ ir sekant diegimo vedlį, įdiegiama „Ubuntu (64-bit)“ operacinė sistema.
\end{enumerate}

\subsubsection{„MacroBase GUI“ analitinio įrankio su grafine naudotojo sąsaja paruošimas}
Sekant „MacroBase“ dokumentacijoje nurodytus žingsnius įdiegiamas „MacroBase GUI“ \cite{macrobase_doc}:
\begin{enumerate}
\item Atidaromas „Ubuntu“ Terminalas;
\item Klonuojama projekto repozitorija:\\\code{git clone https://github.com/stanford-futuredata/macrobase.git}
\item Sukompiliuojamas „MacroBase“ ir paruošiamas darbui:\\\code{cd macrobase; mvn package}
\item Paleidžiamas „MacroBase“ serveris su grafinė sąsaja:\\\code{bin/frontend.sh}
\item Atidarius interneto naršyklę „MacroBase GUI“ pasiekiamas adresu: \\\url{http://localhost:8080}
\end{enumerate}
Dirbant su CSV failas tai yra žingsniai, kurių užtenka parengti „MacroBase GUI“ darbui (Priedas \ref{sec:gui}, pav. \ref{img:gui} ir pav. \ref{img:gui2}).

\subsubsubsection{„MacroBase GUI“ reikalingi papildomi paketai}
Sėkmingam „MacroBase“ diegimo procesui naujai įdiegtoje, „švarioje“ „Ubuntu“ operacinėje sistemoje, reikalingi papildomi pagalbiniai paketai. Reikalingos papildomos „Ubuntu“ terminalo komandos:
\begin{enumerate}
\item Versijavimo kontrolės sistemos „Git“ \cite{git} diegimas:\\\code{sudo apt install git}
\item Java projektų valdymo ir diegimo priemonės „Maven“ \cite{maven} diegimas:\\\code{sudo apt install maven}
\item Java JRE \cite{jre_jdk} diegimas:\\\code{sudo apt-get install default-jre}
\item Java JDK \cite{jre_jdk} diegimas:\\\code{sudo apt-get install default-jdk}
\end{enumerate}

\subsection{Eksperimento vykdymo eiga}
Šiame poskyryje aprašoma eksperimento vykdymo eiga naudojant „Backblaze“ pateiktus 2018 metų pirmo ketvirčio duomenų rinkinius. 
\subsubsection{Eksperimento eiga su „Backblaze“ duomenimis}
Su „Backblaze“ duomenų rinkiniais nagrinėjamas eksperimentas: Diskų, kurie turi didžiausia tikimybę sugesti artimoje ateityje aptikimas. „Backblaze“ naudoja SMART 5, 187, 188, 197 ir 198 (Priedas \ref{sec:smart}) atributus diskų, kurie turi didžiausią tikimybę greitai sugesti numatymui \cite{backblaze_failure}. Eksperimento metu bus analizuojama, kokie diskų modeliai yra labiausiai linkę daryti kritines klaidas ir ar šie duomenys koruliuoja su „Backblaze“ fiksuojamomis klaidomis ir keičiamais diskais.\par
Šio eksperimento metu buvo paimtos pirmos 7 duomenų rinkinio dienos (2018-01-01 - 2018-01-07), iš „Backblaze“ įrašų sąrašo buvo pasirinktas vienas atributas - „Model“, nurodantis disko modelį, ir buvo ištirti tokie atvejai su visomis parinktomis dienomis:
\begin{enumerate}
\item Parinkta metrika: perskirstytų sektorių skaičiaus atributas (\textbf{smart 5 raw});
\item Parinkta metrika: neištaisomų klaidų skaičiaus atributas (\textbf{smart 187 raw});
\item Parinkta metrika: sustabdytų dėl per ilgo darbo laiko komandų skaičiaus atributas (\textbf{smart 188 raw});
\item Parinkta metrika: „nestabilių“ sektorių skaičiaus atributas (\textbf{smart 197 raw});
\item Parinkta metrika: nepataisomų sektorių skaičiaus atributas (\textbf{smart 198 raw});
\item Parinkta metrika: „Backblaze“ klaidą nurodantis skaičiaus atributas (\textbf{failure});
\end{enumerate}
Kiekvieno rezultato duomenimis yra užpildomas atitinkamai pavadintas (pvz.: „failure\_2018-01-01.txt“ ar „188\_2018-01-05.txt“) tekstinis failas.

\subsubsection{Eksperimento eiga su jungtiniais duomenimis}
Su iš „Backblaze“ duomenų rinkinių sukurtais CSV failais buvo nagrinėjami eksperimentai:
\begin{enumerate}
\item Mėnesio dienos kuriomis diskai kaista labiausiai.
\item Darbas su dideliais CSV failais, jų krovimo laikas.
\end{enumerate}

\subsubsubsection{Eksperimentas: mėnesio dienos, kai diskų temperatūros yra neįprastai aukštos}
Apjungtiems mėnesio duomenų CSV failams buvo parinktas datos atributas. Eksperimentui parinkta metrika buvo temperatūra. Atlikti tyrimai su visais trimis mėnesių failais atskirai. Šiuo eksperimentu nagrinėjama ar buvo dienų kuriomis buvo neįprastai aukšta temperatūra ir „MacroBase“ aiškino šia temperatūra datos atributu.


\subsubsubsection{Eksperimentas: darbas su dideliais CSV failais}
Apjungti mėnesių CSV failai ir bendras viso „Backblaze“ ketvirčio CSV failas yra žymaus dydžio failai (apie 800MB ir apie 2,5GB). Mėnesiniai failai susideda iš vidutiniškai 3 mln. įrašų eilučių; bendras failas susideda iš beveik 9 mln įrašų eilučių. Todėl šiuo eksperimentu tiriama ar parengta eksperimentinė aplinka yra pajėgi dirbti su tokio dydžio duomenų rinkiniais, ir kiek laiko su jais užtruks darbas. Eksperimentui parinktas disko modelis kaip atributas. Šiam eksperimentui reikalinga paprasčiausia metrika, todėl metrikai buvo pasirinktas klaidos atributas, kuris gali įgyti tik dvi skirtingas reikšmes: „0“ arba „1“.

\subsection{Eksperimento rezultatų analizė}
Šiame poskyryje aprašyta atliktų eksperimentų su „Backblaze“ 2018 metų pirmo ketvirčio duomenų rinkiniais bei apjungtais pagal mėnesius ir visų duomenų bendrai viename rinkinyje rezultatų analizė.

\subsubsection{Eksperimentų su „Backblaze“ duomenimis gauti rezultatai}
Su „Backblaze“ duomenų rinkiniais buvo atliktas eksperimentas siekiant aptikti diskus, kurie turi didžiausią tikimybę sugesti. Eksperimento metu pasirinkta tirti pirmos duomenų rinkinio savaitės duomenis (2018-01-01 - 2018-01-07). Eksperimento metu surinkti duomenys randami prieduose (Priedas \ref{sec:rez} ir Priedas \ref{sec:rezStats}).\par
Šio eksperimento metu buvo tiriami duomenys pagal atributus, kuriuos „Backblaze“ naudoja prie veikimo pabaigos artėjančių diskų atpažinimui. „MacroBase“ buvo užduota aiškinti metrikas pagal diskų modelio numerius, tokiu būdu gaunama informacija apie didžiausią tikimybę tam tikro modelio diskams, kiekvienos metrikos atžvilgiu, būti anomalijomis.\par

\textbf{Pirmas tyrimas} buvo atliktas su metrika perskirstytų sektorių skaičiumi. Naudojant šią metriką gautos vidutiniškai 708 išskirtys iš vidutiniškai $95 143$ įrašų (\ref{tab:time5} lentelė). Gautuose rezultatuose (\ref{tab:rez5} lentelė) pastebima, jog anomalijas kelia tik „ST10000NM0086“ ir „Hitachi HDS722020ALA330“ diskai. „ST10000NM0086“ modelio diskai yra vidutiniškai turi 13,136537 kartų didesnę tikimybę būti anomalija, o „Hitachi HDS722020ALA330“ diskai net 15,956676 kartų. Net 106 disko modelio „ST10000NM0086“ įrašai yra perskirstytų sektorių anomalijos.\par

\textbf{Antras tyrimas} buvo atliktas naudojant neištaisomų klaidų skaičiaus metriką. Šio tyrimo metu bendras išskirčių skaičius yra mažesnis, nes ne visi diskai turi tyrimui naudotą metrikos atributą. Iš vidutiniškai $68 018$ įrašų 655 yra išskirtys (\ref{tab:time187} lentelė). Guotuose rezultatuos (\ref{tab:rez187} lentelė) matomi trys skirtingi diskų modeliai: „ST4000DM000“, „ST31500541AS“ ir „ST320LT007“. „ST4000DM000“ modelio diskai sukėlė didžiausia kiekį anomalijų, bei turi didžiausią tikimybę būti anomalija - 11,09313357 kartų. Kol kas pasikartojimų tarp pirmo ir antro tyrimo nematome.\par

\textbf{Trečio tyrimo} metrika buvo sustabdymų skaičius dėl per ilgo komandos vykdymo laiko. Šito kaip ir antro tyrimo metu vidutinis įrašų skaičius yra mažesnis, nes ne visi įrašai turi tyrimui naudotą metrikos atributą. Vidutiniškai 2017 išskirčių ir $65 847$ neišskirčių (\ref{tab:time188} lentelė). Šitą anomaliją „MacroBase“ aiškino didesniu skaičiumi modelių nei praeitų tyrimų metu (\ref{tab:rez188} lentelė). Diskai „ST4000DM000“, „ST9250315AS“, „ST31500541AS“ ir „ST320LT007“ buvo labiausiai išskirtos anomalijos. Šio tyrimo metu buvo pamatytas pirmas anomalijų persidengimas: diskai „ST4000DM000“ ir „ST31500541AS“ šio ir praeito tyrimo metu.\par

\textbf{Ketvirtam tyrimui} buvo parinkta „nestabilių“ (laukiančių perskirstymo) sektorių skaičiaus metrika. Šis tyrimas jau vyko su visai rinkinių įrašais - vidutiniškai $95 143$, iš kurių 334 buvo išskirtys (\ref{tab:time197} lentelė). Šiame tyrime matomas (\ref{tab:rez197} lentelė) tas pats „ST4000DM000“ modelis bei pirmame tyrime matytas „Hitachi HDS722020ALA330“ modelis, taip pat du nauji „ST4000DM005“ ir „WDC WD1600AAJS“. Šį kartą „ST4000DM000“ turi tikimybę būti anomaliją ne tokią didelę kaip pirmo tyrimo metu, bet didesnę už trečio - 5,632494857 kartų „ST4000DM000“ modelio diskai turi didesnę tikimybę būti anomalijos.

\textbf{Penktam} ir paskutiniam eksperimento \textbf{tyrimui} naudojant SMART atributus buvo parinktas nepataisomų sektorių skaičiaus metrika. Iš vidutiniškai $95 143$ įrašų 301 buvo išskirtys ir $94 842$ neišskirtys (\ref{tab:time198} lentelė). Šio tyrimo metu (\ref{tab:rez198} lentelė) buvo gauti net trys ankstesnių tyrimų metu gauti modeliai: „ST4000DM000“, „WDC WD1600AAJS“ ir „ST4000DM005“, bei vienas naudas „ST4000DM001“. Iš visų su SMART atributais atliktų tyrimų, galime pastebėti, kad „ST4000DM000“ diskai sukelia daugiausiai anomalijų ir yra nemaža tikimybė jiems sugesti.

Paskutinis atliktas tyrimas buvo su „Backblaze“ klaidos lauku, kuris įgyja reikšmę „1“, jei tai yra paskutinė disko veikimo diena ir diskas yra pakeičiamas nauju. Šio tyrimo duomenys buvo rinkti iki šiol atliktų tyrimų informacijos patikrinimui. Taigi, iš $95 165$ įrašų - vidutiniškai 5 įrašai buvo išskirtys (\ref{tab:timeFail} lentelė). Kaip ir buvo spėta, dominuojantis anomalijų skaičiumi buvo „ST4000DM000“ disko modelis (\ref{tab:rezFail} lentelė). Kaip ir buvo matyti iš pirmų penkių tyrimų „ST4000DM000“ yra tirtosios savaitės labiausiai „Backblaze“ keičiamo modelio diskai ir didžiausią tikimybę gesti ateityje turintis standžiojo disko modelis.

\subsubsection{Eksperimentų su jungtiniais duomenimis gauti rezultatai}
Su dirbtinai sukurtais jungtiniais duomenų rinkiniais buvo atlikti du eksperimentai: išskirtinai aukštos diskų temperatūros „MacroBase“ aiškinamos dienos atributu, ir darbas su dideliais CSV failais.

\subsubsubsection{Dienų, kai diskų temperatūros yra išskirtinai aukštos eksperimento rezultatai}
Pasitelkiant dirbtinai apjungtų duomenų rinkinių sukurtus mėnesių duomenų rinkinių failus (papunktis \ref{subsubsubsec:duomenų_paruošimas}) buvo atliktas tyrimas, kuriuo siekiama išsiaiškinti ar egzistuoja dienos, kai standžiųjų diskų temperatūrų anomalijas „MacroBase“ aiškina pagal datos atributą. Jei temperatūros anomalijos yra aiškinamos datos atributu, yra galimos išorinės diskų fizinės dislokacijos priežastys (pvz.: diskų saugojimo patalpų kondicionavimo sistemos sutrikimai). Eksperimento rezultatai randami prieduose (Priedas \ref{sec:mėnTemp}).\par

2018 metų sausio mėnesio tyrimo metu buvo gautos $36 936$ išimtys ir $3 001 958$ neišimtys (\ref{tab:timeMėn} lentelė). Tačiau tik $15 035$ iš tų išimčių buvo aiškinamos datos atributu. Anomalijos įvyko 2018-01-01 - 2018-01-05 dienomis (\ref{tab:rezMėn} lentelė). Vidutinė tikimybė šitom dienom būti anomalijom yra 2,7281154 kartų didesnė nei vidutinei dienai. Iš gautų duomenų galima spėti, kad pirmą 2018 metų savaitę diskų efektyvumas galėjo nukristi dėl saugojimo patalpų terminių savybių.\par

Kitų dviejų tirtų mėnesių (vasario ir kovo) temperatūros anomalijų skaičiai buvo panašūs į sausio mėnesio rezultatus: $32 298$ išskirtys ir $2 771 455$ neišskirtys vasario mėnesį, $39 455$ išskirtys ir $3 066 805$ neišskirtys kovo mėnesį. Tačiau „MacroBase“ nerado nei vienos anomalijos, kurią paaiškintų data. Tai reiškia, kad nors temperatūros anomalijų skaičius yra didelis, anomalijos tirto laikotarpio metu nepriklausė nuo datos ar aplinkos terminių savybių. Visos temperatūrinės anomalijos vasario ir kovo mėnesiais gali būti paaiškintos pasinaudojus kitus, bet ne datos atributus.

\subsubsubsection{Darbo su dideliais CSV failais eksperimento rezultatai}
Atliekant šį eksperimentą, buvo naudojami tie patys duomenys dideli duomenų rinkiniai. Šiuo eksperimentu buvo siekta ištestuoti: ar „MacroBase“ korektiškai užbaigti analizę duotiems duomenims pateiktoje eksperimento aplinkoje; kiek laiko užtruks paprasčiausio binarinės metrikos analizė. Eksperimento rezultatus randami prieduose (Priedas \ref{sec:big}).\par

Atliekant tyrimą buvo pastėbėtas akivaizdus šuolis laiko atžvilgiu



\sectionnonum{Rezultatai ir išvados}
Rezultatų ir išvadų dalyje turi būti aiškiai išdėstomi pagrindiniai darbo
rezultatai (kažkas išanalizuota, kažkas sukurta, kažkas įdiegta) ir pateikiamos
išvados (daromi nagrinėtų problemų sprendimo metodų palyginimai, teikiamos
rekomendacijos, akcentuojamos naujovės).

\printbibliography[heading=bibintoc]

% \sectionnonum{Sąvokų apibrėžimai}
\sectionnonum{Santrumpos}
\begin{itemize}
\item \textbf{CSV} - „Failo formatas, sudarytas iš Kableliais atskirtų reikšmių“ angl. \textit{Comma-Separated Values}.

\item \textbf{BASH} - „Komandinių eilučių interpretatorius ir komandų kalba“.

\item \textbf{JRE} - „JAVA vykdymo aplinka“ angl. \textit{Java Runtime Environment}.

\item \textbf{JDK} - „JAVA vystymo rinkinys“ angl. \textit{Java Development Kit}.

\item \textbf{S.M.A.R.T} arba \textbf{SMART} - „Save Stebinčios Analizuojančios ir Protokoluojančios Technologijos“ angl. \textit{„Self-Monitoring, Analysis and Reporting Technology“}

\end{itemize}

\appendix

\section{„Backblaze“ duomenų SMART. atributų reikšmės} \label{sec:smart}
„Backblaze“ naudoja „S.M.A.R.T.“ kietųjų diskų protokolavimui. Tačiau „Backblaze“ naudoja tik dalį „S.M.A.R.T.“ atributų:
\begin{itemize}
\item \textit{\textbf{Smart 1 raw, Smart 1 normalized}} - \textbf{Read Error Rate} - (nuo gamintojo priklausanti reikšmė) nuskaitymo klaidų dažnis (mažesnis geresnis).
\item \textit{\textbf{Smart 2 raw, Smart 2 normalized}} - \textbf{Throughput Performance} - bendras disko pralaidumo efektyvumas (didesnis geresnis).
\item \textit{\textbf{Smart 3 raw, Smart 3 normalized}} - \textbf{Spin-Up Time} - vidutinis laikas reikalingas pasiekti maksimalų disko sukimosi greitį (mažesnis geresnis).
\item \textit{\textbf{Smart 4 raw, Smart 4 normalized}} - \textbf{Start/Stop Count} - disko sukimo pradėjimų ir sustabdymų skaičius.
\item \textit{\textbf{Smart 5 raw, Smart 5 normalized}} - \textbf{Reallocated Sectors Count} - perskirstytų sektorių skaičius (mažesnis geresnis). Diskas, kurio nors vienas sektorius bent kartą yra perskirstytas, turi daug didesnę tikimybę sugesti.
\item \textit{\textbf{Smart 7 raw, Smart 7 normalized}} - \textbf{Seek Error Rate} - (nuo gamintojo priklausanti reikšmė) paieškos klaidų dažnis. Paieškos klaida atsiranda, kai įvyksta klaida mechaninėje pozicijos nustatymo sistemoje. Tai gali įvykti dėl įvairių priežasčių, pvz.: pažeidimo servo mechanizme, temperatūros sukelto plėtimosi, kt.
\item \textit{\textbf{Smart 8 raw, Smart 8 normalized}} - \textbf{Seek Time Performance} - vidutinis paieškos efektyvumas (didesnis geresnis). Krintanti reikšmė reiškia kylančias problemas mechaninėje posistemėje.
\item \textit{\textbf{Smart 9 raw, Smart 9 normalized}} - \textbf{Power-On Hours} - disko bendras valandų skaičius, kai diskas yra įjungtas.
\item \textit{\textbf{Smart 10 raw, Smart 10 normalized}} - \textbf{Spin Retry Count} - bandymų įsukti diską skaičius (mažesnis geresnis).
\item \textit{\textbf{Smart 11 raw, Smart 11 normalized}} - \textbf{Recalibration Retries or Calibration Retry Count} - pakartotinių kalibravimų skaičius (mažesnis geresnis).
\item \textit{\textbf{Smart 12 raw, Smart 12 normalized}} - \textbf{Power Cycle Count} - įjungimų ir išjungimų skaičius, žymi disko pilnų įjungimų ir išjungimų skaičių.
\item \textit{\textbf{Smart 13 raw, Smart 13 normalized}} - \textbf{Soft Read Error Rate} - neištaisytos nuskaitymo klaidos apie kurias yra pranešta operaciniai sistemai (mažesnis geresnis)
\item \textit{\textbf{Smart 15 raw, Smart 15 normalized}} - \textit{informacijos apie šį atributą nerasta}.
\item \textit{\textbf{Smart 22 raw, Smart 22 normalized}} - \textbf{Current Helium Level} - (specifinis He8 diskams atributas) matuoja helio kiekį disko viduje.
\item \textit{\textbf{Smart 177 raw, Smart 177 normalized}} - \textbf{Wear Range Delta} - delta tarp labiausiai susidėvėjusių ir mažiausiai susidėvėjusių „Flash“ blokų.
\item \textit{\textbf{Smart 179 raw, Smart 179 normalized}} - \textbf{Used Reserved Block Count Total} - (Samsung diskų naudojamas atributas) - naudojamų rezervuotų blokų bendras skaičius.
\item \textit{\textbf{Smart 181 raw, Smart 181 normalized}} - \textbf{Program Fail Count Total or Non-4K Aligned Access Count} - bendras „Flash“ programos operacijų klaidų skaičius nuo disko naudojimo pradžios (mažesnis geresnis).
\item \textit{\textbf{Smart 182 raw, Smart 182 normalized}} - \textbf{Erase Fail Count} - (Samsung diskų naudojamas atributas) duomenų ištrynimų klaidų skaičius.
\item \textit{\textbf{Smart 183 raw, Smart 183 normalized}} - \textbf{SATA Downshift Error Count or Runtime Bad Block} - (Western digital, Samsung, Seagate atributas) - arba greičio disko sukimo greičio sumažėjimų skaičius arba bendras blokų su darbo metu aptiktomis, bet neištaisytomis klaidomis, skaičius (mažesnis geresnis).
\item \textit{\textbf{Smart 184 raw, Smart 184 normalized}} - \textbf{End-to-End error / IOEDC} - duomenų lygumo (angl. \textit{parity}) klaidų skaičius (mažesnis geresnis).
\item \textit{\textbf{Smart 187 raw, Smart 187 normalized}} - \textbf{Reported Uncorrectable Errors} - skaičius klaidų, kurios negali būti ištaisytos naudojant techninę įrangą (mažesnis geresnis).
\item \textit{\textbf{Smart 188 raw, Smart 188 normalized}} - \textbf{Command Timeout} - skaičius atšauktų operacijų dėl pasibaigusio HDD laiko (angl. \textit{timeout}) (mažesnis geresnis). Paprastai atributo reikšmė lygi „0“.
\item \textit{\textbf{Smart 189 raw, Smart 189 normalized}} - \textbf{High Fly Writes} - rašymo galvutės rašymo per aukštai nuo disko arba ne vietoje klaidų skaičius (mažesnis geresnis).
\item \textit{\textbf{Smart 190 raw, Smart 190 normalized}} - \textbf{Temperature Difference or Airflow Temperature} - reikšmė lygi $100-temperatūra$ °C, gamintojų naudojama nustatyti minimalų slenkstį, kuris atitinka maksimalią temperatūrą.
\item \textit{\textbf{Smart 191 raw, Smart 191 normalized}} - \textbf{G-sense Error Rate} - klaidų skaičius dėl išorinių smūgių ar vibracijos (mažesnis geresnis).
\item \textit{\textbf{Smart 192 raw, Smart 192 normalized}} - \textbf{Power-off Retract Count, Emergency Retract Cycle Count (Fujitsu), or Unsafe Shutdown Count} - nesaugių disko išjungimų skaičius (mažesnis geresnis).
\item \textit{\textbf{Smart 193 raw, Smart 193 normalized}} - \textbf{Load Cycle Count or Load/Unload Cycle Count (Fujitsu)} - skaičius užkrovimų ciklų skaičius (mažesnis geresnis).
\item \textit{\textbf{Smart 194 raw, Smart 194 normalized}} - \textbf{Temperature or Temperature Celsius} - nurodo disko temperatūrą (mažesnė geresnė).
\item \textit{\textbf{Smart 195 raw, Smart 195 normalized}} - \textbf{Hardware ECC Recovered} - (gamintojam specifinis atributas).
\item \textit{\textbf{Smart 196 raw, Smart 196 normalized}} - \textbf{Reallocation Event Count} - pertvarkymo operacijų skaičius (mažesnis geresnis).
\item \textit{\textbf{Smart 197 raw, Smart 197 normalized}} - \textbf{Current Pending Sector Count} - „nestabilių“ sektorių skaičius (mažesnis geresnis).
\item \textit{\textbf{Smart 198 raw, Smart 198 normalized}} - \textbf{(Offline) Uncorrectable Sector Count} - bendras neištaisytų klaidų skaičius skaitant arba rašant sektoriuje (mažesnis geresnis).
\item \textit{\textbf{Smart 199 raw, Smart 199 normalized}} - \textbf{UltraDMA CRC Error Count} - skaičius klaidų perduodant duomenis per laido sąsają (mažesnis geresnis).
\item \textit{\textbf{Smart 200 raw, Smart 200 normalized}} - \textbf{Multi-Zone Error Rate} - rastų klaidų skaičius rašant sektoriuje (mažesnis geresnis).
\item \textit{\textbf{Smart 201 raw, Smart 201 normalized}} - \textbf{Soft Read Error Rate or
TA Counter Detected} - neištaisomų programinės įrangos skaitymo klaidų skaičius (mažesnis geresnis).
\item \textit{\textbf{Smart 222 raw, Smart 222 normalized}} - \textbf{Loaded Hours} - darbo laikas kraunant duomenis (judinant magnetinę galvutę).
\item \textit{\textbf{Smart 223 raw, Smart 223 normalized}} - \textbf{Load/Unload Retry Count} - galvutės pozicijos pakeitimų skaičius.
\item \textit{\textbf{Smart 224 raw, Smart 224 normalized}} - \textbf{Load Friction} - trinties sukeltų pasipriešinimų operuojant skaičius (mažesnis geresnis).
\item \textit{\textbf{Smart 225 raw, Smart 225 normalized}} - \textbf{Load/Unload Cycle Count} - bendras užkrovimų ciklų skaičius (mažesnis geresnis).
\item \textit{\textbf{Smart 226 raw, Smart 226 normalized}} - \textbf{Load 'In'-time} - bendras magnetinės galvutės krovimo laikas.
\item \textit{\textbf{Smart 235 raw, Smart 235 normalized}} - \textbf{Good Block Count AND System(Free) Block Count} - nurodo gerų blokų ir laisvų blokų skaičių.
\item \textit{\textbf{Smart 240 raw, Smart 240 normalized}} - Head Flying Hours or 'Transfer Error Rate' (Fujitsu) - bendras laikas praleista keičiant galvutės poziciją.
\item \textit{\textbf{Smart 241 raw, Smart 241 normalized}} - \textbf{Total LBAs Written} - bendras skaičius rašytų LBA.
\item \textit{\textbf{Smart 242 raw, Smart 242 normalized}} - \textbf{Total LBAs Read} - bendras skaičius skaitytų LBA.
\item \textit{\textbf{Smart 250 raw, Smart 250 normalized}} - \textbf{Read Error Retry Rate} - klaidų skaičius skaitant iš disko (mažesnis geresnis).
\item \textit{\textbf{Smart 251 raw, Smart 251 normalized}} - \textbf{Minimum Spares Remaining} - nurodo likusių nepanaudotų blokų skaičių ir procentinį jų kiekį.
\item \textit{\textbf{Smart 252 raw, Smart 252 normalized}} - \textbf{Newly Added Bad Flash Block} - bendras blogų „Flash“ blokų skaičius nuo pirmo disko palaidimo.
\item \textit{\textbf{Smart 254 raw, Smart 254 normalized}} - \textbf{Free Fall Protection} - skaičius aptiktų „Laisvų Klaidos Įvykių“ (angl. \textit{„Free Fail Events“}.
\item \textit{\textbf{Smart 255 raw, Smart 255 normalized}} - \textit{informacijos apie šį atributą nerasta}.
\end{itemize}

\section{„Backblaze“ CSV failų apjungimo BASH skriptas} \label{sec:skriptas}
BASH skriptas „combine\_csv.sh“ naudotas apjungti „Backblaze“ duomenis į vieną bendrą CSV failą ir 3 atskirus mėnesinius (sausiui, vasariui, kovui) CSV failus:\par

\noindent\code{\#!/bin/bash \\
directory=\$1\\
\\
head -1 \$directory/2018-01-01.csv > combined\_all.csv\\
head -1 \$directory/2018-01-01.csv > combined\_01.csv\\
head -1 \$directory/2018-01-01.csv > combined\_02.csv\\
head -1 \$directory/2018-01-01.csv > combined\_03.csv\\
\\
for file\_name in \$(ls \$directory/*.csv); do sed 1d \$file\_name >> combined\_all.csv; done\\
for file\_name in \$(ls \$directory/2018-01-*.csv); do sed 1d \$file\_name >> combined\_01.csv; done\\
for file\_name in \$(ls \$directory/2018-02-*.csv); do sed 1d \$file\_name >> combined\_02.csv; done\\
for file\_name in \$(ls \$directory/2018-03-*.csv); do sed 1d \$file\_name >> combined\_03.csv; done}\par

„Backblaze“ pateikiami duomenų failai yra pavadinti pagal įrašų rinkimo datą (pvz.: „2018-01-01.csv“).

\section{„MacroBase GUI“ pagrindinis langas} \label{sec:gui}
\begin{figure}[H]
    \centering
    \includegraphics[scale=0.33]{img/gui}
    \caption{„MacroBase GUI“ pagrindinis langas}
    \label{img:gui}
\end{figure}
\begin{figure}[H]
    \centering
    \includegraphics[scale=0.33]{img/gui2}
    \caption{„MacroBase GUI“ pagrindinis langas su pavyzdinių „MacroBase“ duomenų analizės rezultatais}
    \label{img:gui2}
\end{figure}

\section{Diskų gedimų numatymo eksperimento rezultatai} \label{sec:rez}
% tablesgenerator.com - converts calculators (e.g. excel) tables to LaTeX]{beamer}
\begin{table}[H]
\centering
\caption{„Backblaze“ įrašų rinkinių rezultatai pagal SMART 5 metriką.}
\label{tab:rez5}
{\begin{tabular}{|c|l|l|l|l|}
\hline
\textbf{Data}                                   & \multicolumn{1}{c|}{\textbf{Modelis}} & \multicolumn{1}{c|}{\textbf{Santykis}} & \multicolumn{1}{c|}{\textbf{Parama}} & \multicolumn{1}{c|}{\textbf{Įrašai}} \\ \hline
2018-01-01                                      & ST10000NM0086                         & 13,276547                              & 0,150142                             & 106                                  \\ \hline
\cellcolor[HTML]{C0C0C0}{\color[HTML]{333333} } & Hitachi HDS722020ALA330               & 22,368556                              & 0,015581                             & 11                                   \\ \hline
2018-01-02                                      & ST10000NM0086                         & 13,184964                              & 0,149086                             & 106                                  \\ \hline
\cellcolor[HTML]{C0C0C0}                        & Hitachi HDS722020ALA330               & 22,237975                              & 0,015471                             & 11                                   \\ \hline
2018-01-03                                      & ST10000NM0086                         & 13,415309                              & 0,15113                              & 107                                  \\ \hline
\cellcolor[HTML]{C0C0C0}                        & Hitachi HDS722020ALA330               & 22,358395                              & 0,015537                             & 11                                   \\ \hline
2018-01-04                                      & ST10000NM0086                         & 13,407867                              & 0,15113                              & 107                                  \\ \hline
\cellcolor[HTML]{C0C0C0}                        & Hitachi HDS722020ALA330               & 22,346012                              & 0,015537                             & 11                                   \\ \hline
2018-01-05                                      & ST10000NM0086                         & 13,414675                              & 0,14986                              & 107                                  \\ \hline
\cellcolor[HTML]{C0C0C0}                        & Hitachi HDS722020ALA330               & 22,385791                              & 0,015406                             & 11                                   \\ \hline
2018-01-06                                      & ST10000NM0086                         & 11,119056                              & 0,15461                              & 109                                  \\ \hline
2018-01-07                                      & ST10000NM0086                         & 14,137344                              & 0,15461                              & 109                                  \\ \hline
\end{tabular}}
\end{table}

\begin{table}[H]
\centering
\caption{„Backblaze“ įrašų rinkinių rezultatai pagal SMART 187 metriką.}
\label{tab:rez187}
{\begin{tabular}{|l|l|l|l|l|}
\hline
\multicolumn{1}{|c|}{\textbf{Data}} & \multicolumn{1}{c|}{\textbf{Modelis}} & \multicolumn{1}{c|}{\textbf{Santykis}} & \multicolumn{1}{c|}{\textbf{Parama}} & \multicolumn{1}{c|}{\textbf{Įrašai}} \\ \hline
2018-01-01               & ST4000DM000  & 10,731441 & 0,907692 & 590    \\ \hline
\cellcolor[HTML]{C0C0C0} & ST31500541AS & 3,92616   & 0,003077 & 2      \\ \hline
\cellcolor[HTML]{C0C0C0} & ST320LT007   & 9,593051  & 0,001538 & 1      \\ \hline
2018-01-02               & ST4000DM000  & 10,651343 & 0,90687  & 594    \\ \hline
\cellcolor[HTML]{C0C0C0} & ST31500541AS & 3,893089  & 0,003053 & 3      \\ \hline
\cellcolor[HTML]{C0C0C0} & ST320LT007   & 9,521604  & 0,001527 & 1      \\ \hline
2018-01-03               & ST4000DM000  & 10,950332 & 0,90937  & 592    \\ \hline
\cellcolor[HTML]{C0C0C0} & ST31500541AS & 3,932132  & 0,003072 & 2      \\ \hline
\cellcolor[HTML]{C0C0C0} & ST320LT007   & 9,60599   & 0,001536 & 1      \\ \hline
2018-01-04               & ST4000DM000  & 11,202331 & 0,911315 & 596    \\ \hline
\cellcolor[HTML]{C0C0C0} & ST31500541AS & 3,910297  & 0,003058 & 2      \\ \hline
\cellcolor[HTML]{C0C0C0} & ST320LT007   & 9,558811  & 0,001529 & 1      \\ \hline
2018-01-05               & ST4000DM000  & 11,293594 & 0,909786 & 595    \\ \hline
\cellcolor[HTML]{C0C0C0} & ST31500541AS & 3,984939  & 0,003058 & 2      \\ \hline
\cellcolor[HTML]{C0C0C0} & ST320LT007   & 9,720288  & 0,001529 & 1      \\ \hline
2018-01-06               & ST4000DM000  & 11,382906 & 0,907436 & 598    \\ \hline
\cellcolor[HTML]{C0C0C0} & ST31500541AS & 4,047396  & 0,003035 & 2      \\ \hline
\cellcolor[HTML]{C0C0C0} & ST320LT007   & 9,855501  & 0,001517 & 1      \\ \hline
2018-01-07               & ST4000DM000  & 11,439988 & 0,907576 & 599    \\ \hline
\cellcolor[HTML]{C0C0C0} & ST31500541AS & 4,04827   & 0,00303  & 2      \\ \hline
\cellcolor[HTML]{C0C0C0} & ST320LT007   & 9,857412  & 0,001515 & 1      \\ \hline
\end{tabular}}
\end{table}

\begin{table}[H]
\centering
\caption{„Backblaze“ įrašų rinkinių rezultatai pagal SMART 188 metriką.}
\label{tab:rez188}
{\begin{tabular}{|l|l|l|l|l|}
\hline
\multicolumn{1}{|c|}{\textbf{Data}} & \multicolumn{1}{c|}{\textbf{Modelis}} & \multicolumn{1}{c|}{\textbf{Santykis}} & \multicolumn{1}{c|}{\textbf{Parama}} & \multicolumn{1}{c|}{\textbf{Įrašai}} \\ \hline
2018-01-01                          & ST4000DM000                           & 2,944763                               & 0,731426                             & 1585                                 \\ \hline
\cellcolor[HTML]{C0C0C0}            & ST9250315AS                           & 30,817261                              & 0,020305                             & 44                                   \\ \hline
\cellcolor[HTML]{C0C0C0}            & ST31500541AS                          & 6,581815                               & 0,004153                             & 9                                    \\ \hline
\cellcolor[HTML]{C0C0C0}            & ST320LT007                            & 27,368309                              & 0,003692                             & 8                                    \\ \hline
2018-01-02                          & ST4000DM000                           & 2,934509                               & 0,730306                             & 1576                                 \\ \hline
\cellcolor[HTML]{C0C0C0}            & ST9250315AS                           & 30,961801                              & 0,019926                             & 43                                   \\ \hline
\cellcolor[HTML]{C0C0C0}            & ST31500541AS                          & 6,622138                               & 0,004171                             & 9                                    \\ \hline
\cellcolor[HTML]{C0C0C0}            & ST320LT007                            & 27,523551                              & 0,003707                             & 8                                    \\ \hline
2018-01-03                          & ST4000DM000                           & 2,918827                               & 0,729755                             & 1577                                 \\ \hline
\cellcolor[HTML]{C0C0C0}            & ST9250315AS                           & 30,987409                              & 0,020361                             & 44                                   \\ \hline
\cellcolor[HTML]{C0C0C0}            & ST31500541AS                          & 6,620992                               & 0,004165                             & 9                                    \\ \hline
\cellcolor[HTML]{C0C0C0}            & ST320LT007                            & 27,519139                              & 0,003702                             & 8                                    \\ \hline
2018-01-04                          & ST4000DM000                           & 2,94161                                & 0,731494                             & 1591                                 \\ \hline
\cellcolor[HTML]{C0C0C0}            & ST9250315AS                           & 30,798674                              & 0,02023                              & 44                                   \\ \hline
\cellcolor[HTML]{C0C0C0}            & ST31500541AS                          & 6,577933                               & 0,004138                             & 9                                    \\ \hline
\cellcolor[HTML]{C0C0C0}            & ST320LT007                            & 27,353363                              & 0,003678                             & 8                                    \\ \hline
2018-01-05                          & ST4000DM000                           & 3,019693                               & 0,731282                             & 1592                                 \\ \hline
\cellcolor[HTML]{C0C0C0}            & ST9250315AS                           & 31,204788                              & 0,020211                             & 44                                   \\ \hline
\cellcolor[HTML]{C0C0C0}            & ST31500541AS                          & 6,672341                               & 0,004134                             & 9                                    \\ \hline
\cellcolor[HTML]{C0C0C0}            & ST320LT007                            & 27,71684                               & 0,003675                             & 8                                    \\ \hline
2018-01-06                          & ST4000DM000                           & 3,210627                               & 0,736015                             & 1592                                 \\ \hline
\cellcolor[HTML]{C0C0C0}            & ST9250315AS                           & 32,021745                              & 0,020342                             & 44                                   \\ \hline
\cellcolor[HTML]{C0C0C0}            & ST31500541AS                          & 6,861249                               & 0,004161                             & 9                                    \\ \hline
\cellcolor[HTML]{C0C0C0}            & ST320LT007                            & 28,444145                              & 0,003699                             & 8                                    \\ \hline
2018-01-07                          & ST4000DM000                           & 3,063261                               & 0,726277                             & 1592                                 \\ \hline
\cellcolor[HTML]{C0C0C0}            & ST9250315AS                           & 31,642066                              & 0,020073                             & 44                                   \\ \hline
\cellcolor[HTML]{C0C0C0}            & ST31500541AS                          & 6,774701                               & 0,004106                             & 9                                    \\ \hline
\cellcolor[HTML]{C0C0C0}            & ST320LT007                            & 28,11094                               & 0,00365                              & 8                                    \\ \hline
\end{tabular}}
\end{table}

\begin{table}[H]
\centering
\caption{„Backblaze“ įrašų rinkinių rezultatai pagal SMART 197 metriką.}
\label{tab:rez197}
{\begin{tabular}{|l|l|l|l|l|}
\hline
\multicolumn{1}{|c|}{\textbf{Data}} & \multicolumn{1}{c|}{\textbf{Modelis}} & \multicolumn{1}{c|}{\textbf{Santykis}} & \multicolumn{1}{c|}{\textbf{Parama}} & \multicolumn{1}{c|}{\textbf{Įrašai}} \\ \hline
2018-01-01                          & ST4000DM000                           & 5,581127                               & 0,745455                             & 246                                  \\ \hline
\cellcolor[HTML]{C0C0C0}            & ST4000DM005                           & 2,785014                               & 0,00303                              & 1                                    \\ \hline
\cellcolor[HTML]{C0C0C0}            & Hitachi HDS722020ALA330               & 2,416696                               & 0,00303                              & 1                                    \\ \hline
\cellcolor[HTML]{C0C0C0}            & WDC WD1600AAJS                        & 5,179099                               & 0,00303                              & 1                                    \\ \hline
2018-01-02                          & ST4000DM000                           & 5,615303                               & 0,746224                             & 247                                  \\ \hline
\cellcolor[HTML]{C0C0C0}            & ST4000DM005                           & 2,777484                               & 0,003021                             & 1                                    \\ \hline
\cellcolor[HTML]{C0C0C0}            & Hitachi HDS722020ALA330               & 2,409746                               & 0,003021                             & 1                                    \\ \hline
\cellcolor[HTML]{C0C0C0}            & WDC WD1600AAJS                        & 4,826323                               & 0,003021                             & 1                                    \\ \hline
2018-01-03                          & ST4000DM000                           & 5,718268                               & 0,75                                 & 249                                  \\ \hline
\cellcolor[HTML]{C0C0C0}            & ST4000DM005                           & 2,689523                               & 0,003012                             & 1                                    \\ \hline
\cellcolor[HTML]{C0C0C0}            & Hitachi HDS722020ALA330               & 2,400746                               & 0,003012                             & 1                                    \\ \hline
\cellcolor[HTML]{C0C0C0}            & WDC WD1600AAJS                        & 4,81239                                & 0,003012                             & 1                                    \\ \hline
2018-01-04                          & ST4000DM000                           & 5,679619                               & 0,749245                             & 248                                  \\ \hline
\cellcolor[HTML]{C0C0C0}            & ST4000DM005                           & 2,625275                               & 0,003021                             & 1                                    \\ \hline
\cellcolor[HTML]{C0C0C0}            & Hitachi HDS722020ALA330               & 2,411657                               & 0,003021                             & 1                                    \\ \hline
\cellcolor[HTML]{C0C0C0}            & WDC WD1600AAJS                        & 4,829281                               & 0,003021                             & 1                                    \\ \hline
2018-01-05                          & ST4000DM000                           & 5,72532                                & 0,747748                             & 249                                  \\ \hline
\cellcolor[HTML]{C0C0C0}            & ST4000DM005                           & 2,644973                               & 0,003003                             & 1                                    \\ \hline
\cellcolor[HTML]{C0C0C0}            & Hitachi HDS722020ALA330               & 2,430448                               & 0,003003                             & 1                                    \\ \hline
\cellcolor[HTML]{C0C0C0}            & WDC WD1600AAJS                        & 4,858351                               & 0,003003                             & 1                                    \\ \hline
2018-01-06                          & ST4000DM000                           & 5,666128                               & 0,741935                             & 253                                  \\ \hline
\cellcolor[HTML]{C0C0C0}            & ST4000DM005                           & 2,597019                               & 0,002933                             & 1                                    \\ \hline
\cellcolor[HTML]{C0C0C0}            & WDC WD1600AAJS                        & 4,787528                               & 0,002933                             & 1                                    \\ \hline
2018-01-07                          & ST4000DM000                           & 5,441699                               & 0,733918                             & 251                                  \\ \hline
\cellcolor[HTML]{C0C0C0}            & ST4000DM005                           & 2,589356                               & 0,002924                             & 1                                    \\ \hline
\cellcolor[HTML]{C0C0C0}            & WDC WD1600AAJS                        & 4,776212                               & 0,002924                             & 1                                    \\ \hline
\end{tabular}}
\end{table}

\begin{table}[H]
\centering
\caption{„Backblaze“ įrašų rinkinių rezultatai pagal SMART 198 metriką.}
\label{tab:rez198}
{\begin{tabular}{|l|l|l|l|l|}
\hline
\multicolumn{1}{|c|}{\textbf{Data}} & \multicolumn{1}{c|}{\textbf{Modelis}} & \multicolumn{1}{c|}{\textbf{Santykis}} & \multicolumn{1}{c|}{\textbf{Parama}} & \multicolumn{1}{c|}{\textbf{Įrašai}} \\ \hline
2018-01-01                          & ST4000DM000                           & 9,2748                                 & 0,828283                             & 246                                  \\ \hline
\cellcolor[HTML]{C0C0C0}            & WDC WD1600AAJS                        & 5,976668                               & 0,003367                             & 1                                    \\ \hline
\cellcolor[HTML]{C0C0C0}            & ST4000DM005                           & 3,316613                               & 0,003367                             & 1                                    \\ \hline
\cellcolor[HTML]{C0C0C0}            & ST4000DM001                           & 2,271584                               & 0,013468                             & 4                                    \\ \hline
2018-01-02                          & ST4000DM000                           & 9,330921                               & 0,828859                             & 247                                  \\ \hline
\cellcolor[HTML]{C0C0C0}            & WDC WD1600AAJS                        & 5,582143                               & 0,003356                             & 1                                    \\ \hline
\cellcolor[HTML]{C0C0C0}            & ST4000DM005                           & 3,306457                               & 0,003356                             & 1                                    \\ \hline
\cellcolor[HTML]{C0C0C0}            & ST4000DM001                           & 2,265228                               & 0,013423                             & 4                                    \\ \hline
2018-01-03                          & ST4000DM000                           & 9,575818                               & 0,832776                             & 249                                  \\ \hline
\cellcolor[HTML]{C0C0C0}            & WDC WD1600AAJS                        & 5,564131                               & 0,003344                             & 1                                    \\ \hline
\cellcolor[HTML]{C0C0C0}            & ST4000DM005                           & 3,207011                               & 0,003344                             & 1                                    \\ \hline
\cellcolor[HTML]{C0C0C0}            & ST4000DM001                           & 2,257359                               & 0,013378                             & 4                                    \\ \hline
2018-01-04                          & ST4000DM000                           & 9,708536                               & 0,835017                             & 248                                  \\ \hline
\cellcolor[HTML]{C0C0C0}            & WDC WD1600AAJS                        & 5,610976                               & 0,003367                             & 1                                    \\ \hline
\cellcolor[HTML]{C0C0C0}            & ST4000DM005                           & 3,154697                               & 0,003367                             & 1                                    \\ \hline
\cellcolor[HTML]{C0C0C0}            & ST4000DM001                           & 2,320308                               & 0,013468                             & 4                                    \\ \hline
2018-01-05                          & ST4000DM000                           & 9,704559                               & 0,832776                             & 249                                  \\ \hline
\cellcolor[HTML]{C0C0C0}            & WDC WD1600AAJS                        & 5,638116                               & 0,003344                             & 1                                    \\ \hline
\cellcolor[HTML]{C0C0C0}            & ST4000DM005                           & 3,17309                                & 0,003344                             & 1                                    \\ \hline
\cellcolor[HTML]{C0C0C0}            & ST4000DM001                           & 2,332064                               & 0,013378                             & 4                                    \\ \hline
2018-01-06                          & ST4000DM000                           & 9,141099                               & 0,821429                             & 253                                  \\ \hline
\cellcolor[HTML]{C0C0C0}            & ST4000DM001                           & 2,27726                                & 0,012987                             & 4                                    \\ \hline
\cellcolor[HTML]{C0C0C0}            & WDC WD1600AAJS                        & 5,514558                               & 0,003247                             & 1                                    \\ \hline
\cellcolor[HTML]{C0C0C0}            & ST4000DM005                           & 3,089422                               & 0,003247                             & 1                                    \\ \hline
2018-01-07                          & ST4000DM000                           & 8,762776                               & 0,814935                             & 251                                  \\ \hline
\cellcolor[HTML]{C0C0C0}            & ST4000DM001                           & 2,281408                               & 0,012987                             & 4                                    \\ \hline
\cellcolor[HTML]{C0C0C0}            & WDC WD1600AAJS                        & 5,52402                                & 0,003247                             & 1                                    \\ \hline
\cellcolor[HTML]{C0C0C0}            & ST4000DM005                           & 3,095831                               & 0,003247                             & 1                                    \\ \hline
\end{tabular}}
\end{table}

\begin{table}[H]
\centering
\caption{„Backblaze“ įrašų rinkinių rezultatai pagal „Failure“ metriką.}
\label{tab:rezFail}
{\begin{tabular}{|l|l|l|l|l|}
\hline
\multicolumn{1}{|c|}{\textbf{Data}} & \multicolumn{1}{c|}{\textbf{Modelis}} & \multicolumn{1}{c|}{\textbf{Santykis}} & \multicolumn{1}{c|}{\textbf{Parama}} & \multicolumn{1}{c|}{\textbf{Įrašai}} \\ \hline
2018-01-01                          & ST4000DM000                           & 2,48423                                & 0,666667                             & 2                                    \\ \hline
\cellcolor[HTML]{C0C0C0}            & HGST HUH728080ALE600                  & 42,566018                              & 0,333333                             & 1                                    \\ \hline
2018-01-03                          & ST4000DM000                           & 8,846895                               & 0,833333                             & 5                                    \\ \hline
2018-01-04                          & ST4000DM000                           & Infinity                               & 1                                    & 4                                    \\ \hline
2018-01-05                          & ST12000NM0007                         & 3,758622                               & 0,333333                             & 2                                    \\ \hline
2018-01-06                          & ST4000DM000                           & 4,888753                               & 0,75                                 & 3                                    \\ \hline
2018-01-07                          & ST4000DM000                           & Infinity                               & 1                                    & 2                                    \\ \hline
\end{tabular}}
\end{table}

\section{Diskų gedimų numatymo eksperimento rezultatų statistikos} \label{sec:rezStats}

\begin{table}[H]
\centering
\caption{„Backblaze“ įrašų rinkinių analizių statistikos pagal SMART 5 metriką.}
\label{tab:time5}
{\begin{tabular}{|l|l|l|l|l|l|l|l|}
\hline
\textbf{Data}                 & ...-01	   & ...-02     & ...-03     & ...-04 	  & ...-05     & ...-06     & ...-07     \\ \hline
\textbf{Išskirtys}            & 706        & 711        & 708        & 708        & 714        & 705        & 705        \\ \hline
\textbf{Neišskirtys}          & 93480      & 93612      & 93707      & 93656      & 94627      & 95921      & 96043      \\ \hline
\textbf{Krovimo Laikas}       & 1186ms     & 1390ms     & 1576ms     & 1316ms     & 1890ms     & 1558ms     & 1654ms     \\ \hline
\textbf{Vykdymo Laikas}       & 180ms      & 220ms      & 245ms      & 238ms      & 418ms      & 255ms      & 214ms      \\ \hline
\textbf{Apibendrinimo Laikas} & 40ms       & 75ms       & 43ms       & 82ms       & 43ms       & 43ms       & 46ms       \\ \hline
\end{tabular}}
\end{table}

\begin{table}[H]
\centering
\caption{„Backblaze“ įrašų rinkinių analizių statistikos pagal SMART 187 metriką.}
\label{tab:time187}
\begin{tabular}{|l|l|l|l|l|l|l|l|}
\hline
\textbf{Data}                 & ...-01	   & ...-02     & ...-03     & ...-04 	  & ...-05     & ...-06     & ...-07     \\ \hline
\textbf{Išskirtys}            & 650        & 655        & 651        & 654        & 654        & 659        & 660        \\ \hline
\textbf{Neišskirtys}          & 66406      & 66497      & 66584      & 66614      & 67563      & 68881      & 68997      \\ \hline
\textbf{Krovimo Laikas}       & 1506ms     & 2050ms     & 1747ms     & 1412ms     & 1220ms     & 1903ms     & 1633ms     \\ \hline
\textbf{Vykdymo Laikas}       & 196ms      & 314ms      & 200ms      & 164ms      & 160ms      & 288ms      & 229ms      \\ \hline
\textbf{Apibendrinimo Laikas} & 30ms       & 32ms       & 28ms       & 22ms       & 24ms       & 37ms       & 27ms       \\ \hline
\end{tabular}
\end{table}

\begin{table}[H]
\caption{„Backblaze“ įrašų rinkinių analizių statistikos pagal SMART 188 metriką.}
\label{tab:time188}
{\begin{tabular}{|l|l|l|l|l|l|l|l|}
\hline
\textbf{Data}                 & ...-01	   & ...-02     & ...-03     & ...-04 	  & ...-05     & ...-06     & ...-07     \\ \hline
\textbf{Išskirtys}            & 2167       & 2158       & 2161       & 2175       & 2177       & 2163       & 2192       \\ \hline
\textbf{Neišskirtys}          & 64889      & 64994      & 65074      & 65093      & 66040      & 67377      & 67465      \\ \hline
\textbf{Krovimo Laikas}       & 2071ms     & 1622ms     & 1747ms     & 1996ms     & 1936ms     & 1683ms     & 1514ms     \\ \hline
\textbf{Vykdymo Laikas}       & 427ms      & 288ms      & 226ms      & 199ms      & 222ms      & 293ms      & 234ms      \\ \hline
\textbf{Apibendrinimo Laikas} & 38ms       & 46ms       & 34ms       & 33ms       & 39ms       & 48ms       & 34ms       \\ \hline
\end{tabular}}
\end{table}

\begin{table}[H]
\centering
\caption{„Backblaze“ įrašų rinkinių analizių statistikos pagal SMART 197 metriką.}
\label{tab:time197}
{\begin{tabular}{|l|l|l|l|l|l|l|l|}
\hline
\textbf{Data}                 & ...-01	   & ...-02     & ...-03     & ...-04 	  & ...-05     & ...-06     & ...-07     \\ \hline
\textbf{Išskirtys}            & 330        & 331        & 332        & 331        & 333        & 341        & 342        \\ \hline
\textbf{Neišskirtys}          & 93856      & 93992      & 94083      & 94033      & 95008      & 96285      & 96406      \\ \hline
\textbf{Krovimo Laikas}       & 2441ms     & 1758ms     & 2111ms     & 2337ms     & 1471ms     & 1565ms     & 1584ms     \\ \hline
\textbf{Vykdymo Laikas}       & 255ms      & 296ms      & 252ms      & 249ms      & 217ms      & 242ms      & 276ms      \\ \hline
\textbf{Apibendrinimo Laikas} & 85ms       & 47ms       & 42ms       & 48ms       & 65ms       & 41ms       & 44ms       \\ \hline
\end{tabular}}
\end{table}

\begin{table}[H]
\centering
\caption{„Backblaze“ įrašų rinkinių analizių statistikos pagal SMART 198 metriką.}
\label{tab:time198}
{\begin{tabular}{|l|l|l|l|l|l|l|l|}
\hline
\textbf{Data}                 & ...-01	   & ...-02     & ...-03     & ...-04 	  & ...-05     & ...-06     & ...-07     \\ \hline
\textbf{Išskirtys}            & 297        & 298        & 299        & 297        & 299        & 308        & 308        \\ \hline
\textbf{Neišskirtys}          & 93889      & 94025      & 94116      & 94067      & 95042      & 96318      & 96440      \\ \hline
\textbf{Krovimo Laikas}       & 1572ms     & 1911ms     & 2244ms     & 3161ms     & 1597ms     & 1602ms     & 1704ms     \\ \hline
\textbf{Vykdymo Laikas}       & 354ms      & 254ms      & 288ms      & 279ms      & 229ms      & 214ms      & 229ms      \\ \hline
\textbf{Apibendrinimo Laikas} & 43ms       & 45ms       & 56ms       & 41ms       & 51ms       & 42ms       & 44ms       \\ \hline
\end{tabular}}
\end{table}

\begin{table}[H]
\centering
\caption{„Backblaze“ įrašų rinkinių analizių statistikos pagal „Failure“ metriką.}
\label{tab:timeFail}
{\begin{tabular}{|l|l|l|l|l|l|l|l|}
\hline
\textbf{Data}                 & ...-01	   & ...-02     & ...-03     & ...-04 	  & ...-05     & ...-06     & ...-07     \\ \hline
\textbf{Išskirtys}            & 3          & 8          & 6          & 4          & 6          & 4          & 2          \\ \hline
\textbf{Neišskirtys}          & 94183      & 94315      & 94409      & 94360      & 95522      & 96622      & 96746      \\ \hline
\textbf{Krovimo Laikas}       & 1643ms     & 2100ms     & 1730ms     & 2270ms     & 1702ms     & 2157ms     & 2258ms     \\ \hline
\textbf{Vykdymo Laikas}       & 252ms      & 355ms      & 223ms      & 221ms      & 239ms      & 255ms      & 250ms      \\ \hline
\textbf{Apibendrinimo Laikas} & 46ms       & 57ms       & 38ms       & 41ms       & 41ms       & 47ms       & 49ms       \\ \hline
\end{tabular}}
\end{table}

\begin{table}[H]
\centering
\caption{„Backblaze“ įrašų rinkinių analizių statistikų vidurkiai pagal tirtas metrikas (SMART metrikos žymimos tik skaičiumi).}
\label{tab:timeAvg}
{\begin{tabular}{|l|l|l|l|l|l|l|}
\hline
\textbf{Metrika} 		   & 5        & 187       & 188       & 197       & 198       & failure  \\ \hline
\textbf{Vid. Išskirtys}    & 708,14   & 654,71    & 2170,43   & 334,29    & 300,86    & 4,71     \\ \hline
\textbf{Vid. Neišskirtys}  & 94435,14 & 67363,14  & 65847,43  & 94809     & 94842,43  & 95165,29 \\ \hline
\textbf{Vid. Krov. Laikas} & 1510ms   & 1638,71ms & 1795,57ms & 1895,29ms & 1970,14ms & 1980ms   \\ \hline
\textbf{Vid. Vykd. Laikas} & 252,86ms & 221,57ms  & 269,86ms  & 255,29ms  & 263,86ms  & 256,43ms \\ \hline
\textbf{Vid. Apib. Laikas} & 53,14ms  & 28,57ms   & 38,86ms   & 53,14ms   & 46ms      & 45,57ms  \\ \hline
\end{tabular}}
\end{table}

\section{Mėnesio temperatūrų eksperimento rezultatai ir statistikos} \label{sec:mėnTemp}
\begin{table}[H]
\centering
\caption{Datos, kurių aukštos diskų temperatūrų anomalijas „MacroBase“ aiškina datos atributu.}
\label{tab:rezMėn}
{\begin{tabular}{|l|l|l|l|l|}
\hline
\multicolumn{1}{|c|}{\textbf{Mėnesis}} & \multicolumn{1}{c|}{\textbf{Data}} & \multicolumn{1}{c|}{\textbf{Santykis}} & \multicolumn{1}{c|}{\textbf{Parama}} & \multicolumn{1}{c|}{\textbf{Įrašai}} \\ \hline
2018-01                                & 2018-01-01                         & 2,287333                               & 0,069255                             & 2558                                 \\ \hline
\cellcolor[HTML]{C0C0C0}               & 2018-01-02                         & 2,248474                               & 0,06828                              & 2522                                 \\ \hline
\cellcolor[HTML]{C0C0C0}               & 2018-01-03                         & 2,803709                               & 0,083442                             & 3082                                 \\ \hline
\cellcolor[HTML]{C0C0C0}               & 2018-01-04                         & 3,158734                               & 0,092809                             & 3428                                 \\ \hline
\cellcolor[HTML]{C0C0C0}               & 2018-01-05                         & 3,142327                               & 0,093269                             & 3445                                 \\ \hline
\end{tabular}}
\end{table}

\begin{table}[H]
\centering
\caption{Kiekvieno mėnesio temperatūros anomalijų analizavimo aiškinant jas pagal datos atributą statistikos.}
\label{tab:timeMėn}
{\begin{tabular}{|l|l|l|l|}
\hline
\textbf{Mėnesis}              & 2018-01 & 2018-02 & 2018-03 \\ \hline
\textbf{Išskirtys}            & 36936   & 32298   & 39455   \\ \hline
\textbf{Neišskirtys}         & 3001958 & 2771455 & 3066805 \\ \hline
\textbf{Krovimo Laikas}       & 27958ms & 25393ms & 42009ms \\ \hline
\textbf{Vykdymo Laikas}       & 9231ms  & 6447ms  & 14366ms \\ \hline
\textbf{Apibendrinimo Laikas} & 1831ms  & 2025ms  & 2477ms  \\ \hline
\end{tabular}}
\end{table}

\section{Didelių failų eksperimento rezultatai ir statistikos} \label{sec:big}

\begin{table}[H]
\centering
\caption{„MacroBase“ didelių duomenų rinkinių analizavimo rezultatai.}
\label{tab:rezBig}
{\begin{tabular}{|l|l|l|l|l|}
\hline
\multicolumn{1}{|c|}{\textbf{Failas}} & \multicolumn{1}{c|}{\textbf{Modelis}} & \multicolumn{1}{c|}{\textbf{Santykis}} & \multicolumn{1}{c|}{\textbf{Parama}} & \multicolumn{1}{c|}{\textbf{Įrašai}} \\ \hline
combined\_01.csv                      & ST4000DM000                           & 2,814524                               & 0,592                                & 74                                   \\ \hline
\cellcolor[HTML]{C0C0C0}              & Samsung SSD 850 EVO 1TB               & 137,262013                             & 0,008                                & 1                                    \\ \hline
\cellcolor[HTML]{C0C0C0}              & TOSHIBA MQ01ABF050                    & 8,965289                               & 0,048                                & 6                                    \\ \hline
\cellcolor[HTML]{C0C0C0}              & ST4000DM005                           & 10,798378                              & 0,008                                & 1                                    \\ \hline
combined\_02.csv                      & ST4000DM000                           & 2,016884                               & 0,508065                             & 63                                   \\ \hline
\cellcolor[HTML]{C0C0C0}              & HGST HUS726040ALE610                  & 52,902965                              & 0,016129                             & 2                                    \\ \hline
\cellcolor[HTML]{C0C0C0}              & WDC WD5000LPVX                        & 3,889555                               & 0,016129                             & 2                                    \\ \hline
\cellcolor[HTML]{C0C0C0}              & ST500LM012 HN                         & 5,335109                               & 0,040323                             & 5                                    \\ \hline
\cellcolor[HTML]{C0C0C0}              & Samsung SSD 850 EVO 1TB               & 3782,61752                             & 0,072581                             & 9                                    \\ \hline
\cellcolor[HTML]{C0C0C0}              & WDC WD1600AAJS                        & 18,791637                              & 0,008065                             & 1                                    \\ \hline
combined\_03.csv                      & TOSHIBA MQ01ABF050                    & 2,977068                               & 0,022989                             & 2                                    \\ \hline
\cellcolor[HTML]{C0C0C0}              & TOSHIBA MQ01ABF050M                   & 3,434047                               & 0,011494                             & 1                                    \\ \hline
\cellcolor[HTML]{C0C0C0}              & WDC WD5000LPVX                        & 6,295184                               & 0,022989                             & 2                                    \\ \hline
\cellcolor[HTML]{C0C0C0}              & WDC WD30EFRX                          & 4,461698                               & 0,011494                             & 1                                    \\ \hline
\cellcolor[HTML]{C0C0C0}              & ST4000DM005                           & 15,221751                              & 0,011494                             & 1                                    \\ \hline
\cellcolor[HTML]{C0C0C0}              & Hitachi HDS5C3030ALA630               & 5158,159987                            & 0,011494                             & 1                                    \\ \hline
\cellcolor[HTML]{C0C0C0}              & Hitachi HDS722020ALA330               & 243,448273                             & 0,057471                             & 5                                    \\ \hline
\cellcolor[HTML]{C0C0C0}              & ST4000DX002                           & 1385,107938                            & 0,045977                             & 4                                    \\ \hline
\end{tabular}}
\end{table}

\begin{table}[H]
\centering
\caption{„MacroBase“ didelių duomenų rinkinių analizavimo statistikos.}
\label{my-label}
{\begin{tabular}{|l|l|l|l|l|}
\hline
\textbf{Failo pavadinimas}            & combined\_01.csv & combined\_02.csv & combined\_03.csv & combined\_all.csv           \\ \hline
\textbf{Failo dydis}                  & 846MB            & 782,3MB          & 868,2MB          & 2,5GB                       \\ \hline
\textbf{Failo įrašų eilučių skaičius} & 3 039 306        & 2 803 852        & 3 106 334        & 8 949 492                   \\ \hline
\textbf{Išskirtys}                    & 125              & 124              & 87               & \cellcolor[HTML]{EFEFEF}N/A \\ \hline
\textbf{Neišskirtys}                  & 3039181          & 2803728          & 3106247          & \cellcolor[HTML]{EFEFEF}N/A \\ \hline
\textbf{Krovimo Laikas}               & 32648ms          & 30286ms          & 41892ms          & \cellcolor[HTML]{EFEFEF}N/A \\ \hline
\textbf{Vykdymo Laikas}               & 8896ms           & 5177ms           & 10558ms          & \cellcolor[HTML]{EFEFEF}N/A \\ \hline
\textbf{Apibendrinimo Laikas}         & 1789ms           & 1726ms           & 2180ms           & \cellcolor[HTML]{EFEFEF}N/A \\ \hline
\end{tabular}}
\end{table}

\end{document}
